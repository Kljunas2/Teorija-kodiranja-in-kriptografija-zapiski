\documentclass[a3paper,8pt]{extarticle}
% \usepackage[utf8]{inputenc}
\usepackage[mathletters]{ucs}
\usepackage[utf8x]{inputenc}

\usepackage{fancyhdr}

\usepackage[pdftex]{graphicx} % Required for including pictures
\usepackage[pdftex,linkcolor=black,pdfborder={0 0 0}]{hyperref} % Format links for pdf
\usepackage{calc} % To reset the counter in the document after title page
\usepackage{enumitem} % Includes lists

\usepackage{textcomp}
\usepackage{eurosym}

\usepackage{ dsfont } % font za množice
% tabele
\usepackage{array}
\usepackage{wrapfig}

\usepackage{tikz,forest}
\usetikzlibrary{arrows.meta}

\frenchspacing % No double spacing between sentences
\setlength{\parindent}{0pt}
\setlength{\parskip}{0.3em}

\usepackage{mathtools}
\usepackage{blkarray, bigstrut} %


\usepackage{amssymb,amsmath,amsthm,amsfonts}
\usepackage{multicol,multirow}
\usepackage{calc}
\usepackage{ifthen}
\usepackage{tabularx}
\usepackage[landscape]{geometry}
\usepackage{listings}
\usepackage{inconsolata}
%\usepackage[colorlinks=true,citecolor=blue,linkcolor=blue]{hyperref}
%\usepackage{accents}

\newcommand{\vect}[1]{\accentset{\rightharpoonup}{#1}}

\ifthenelse{\lengthtest { \paperwidth = 11in}}
    { \geometry{top=.5in,left=.5in,right=.5in,bottom=.5in} }
	{\ifthenelse{ \lengthtest{ \paperwidth = 297mm}}
		{\geometry{top=1cm,left=1cm,right=1cm,bottom=1cm} }
		{\geometry{top=1cm,left=1cm,right=1cm,bottom=1cm} }
	}
\pagestyle{empty}
\makeatletter
\renewcommand{\section}{\@startsection{section}{1}{0mm}%
                                {-1ex plus -.5ex minus -.2ex}%
                                {0.5ex plus .2ex}%x
                                {\normalfont\large\bfseries}}
\renewcommand{\subsection}{\@startsection{subsection}{2}{0mm}%
                                {-1explus -.5ex minus -.2ex}%
                                {0.5ex plus .2ex}%
                                {\normalfont\normalsize\bfseries}}
\renewcommand{\subsubsection}{\@startsection{subsubsection}{3}{0mm}%
                                {-1ex plus -.5ex minus -.2ex}%
                                {1ex plus .2ex}%
                                {\normalfont\small\bfseries}}
\makeatother
\setcounter{secnumdepth}{0}
%\setlength{\parindent}{0pt}
%\setlength{\parskip}{0pt plus 0.5ex}

% listings okolje za psevdo kodo
\lstnewenvironment{koda}[1][] %defines the algorithm listing environment
{   
    \lstset{ %this is the stype
        mathescape=true,
        basicstyle=\scriptsize, 
		columns=flexible,
        keywordstyle=\bfseries\em,
        keywords={,vhod, izhod, zacetek, konec, koncamo, ponavljaj, dokler, ce, vrni, za, vsak, vse, v, sicer,} %add the keywords you want, or load a language as Rubens explains in his comment above.
        xleftmargin=.1\textwidth,
		tabsize=4,
		%frame=leftline,xleftmargin=5pt,xrightmargin=5pt,framesep=5pt,
		%inputencoding = utf8,
		extendedchars = true,
		literate={ž}{{\ˇz}}1 {š}{{\ˇs}}1 {č}{{\ˇc}}1 {Ž}{{\ˇZ}}1 {Š}{{\ˇS}}1 {Č}{{\ˇC}}1,
        #1 % this is to add specific settings to an usage of this environment (for instnce, the caption and referable label)
    }
}
{}
% -----------------------------------------------------------------------

\begin{document} 

\begin{multicols}{4}
\setlength{\premulticols}{1pt}
\setlength{\postmulticols}{1pt}
\setlength{\multicolsep}{1pt}
\setlength{\columnsep}{2pt}

\section*{Kriptosistem}
\begin{align*}
	\mathcal{B} &\dots \text{besedila} \\
	\mathcal{C} &\dots \text{kriptogrami} \\
	\mathcal{K} &\dots \text{ključi} \\
	\mathcal{E} = \{E_k : \mathcal{B} \to \mathcal{C}; k \in \mathcal{K} \} &\dots \text{kodirne f.} \\
	\mathcal{D} = \{D_k : \mathcal{C} \to \mathcal{B}; k \in \mathcal{K} \} &\dots \text{dekodirne f.} \\
\end{align*}
Za vsak $e \in \mathcal{K}$ obstaja $d \in \mathcal{K}$
\[ D_d(E_e(x)) = x \quad \forall x \in \mathcal{B}\]

Vsaka kodrirna funkcija $E_k \in \mathcal{E}$ je injektivna.

\section*{Klasični kriptosistem}
\subsection*{Cezarjeva šifra}
\[ \mathcal{B} = \mathcal{C} = \mathcal{K} = \mathbb{Z}_{25}\]
\[ E_k(x) \equiv x + k \mod 25\]
\[ D_k(y) \equiv y - k \mod 25\]

\subsection*{Substitucijska šifra}
\[ \mathcal{B} = \mathcal{C} = \mathbb{Z}_{25}, \quad \mathcal{K} = S(\mathbb{Z}_{25})\]
Ključ je permutacija $\pi \in \mathcal{K}$
\[ E_k(x) = \pi(x) \]
\[ D_k(y) = \pi^{-1}(y) \]

\subsection*{Afina šifra}
\[ \mathcal{B} = \mathcal{C} = \mathbb{Z}_{25}, \quad \mathcal{K} = \mathbb{Z}_{25}^{*} \times \mathbb{Z}_{25} \]
Ključ $(a, b) \in \mathcal{K}$
\[ K_{(a,b)}(x) = ax + b \mod 25\]
\[ D_{(a,b)}(y) = a^{-1}(y - b) \mod 25\]

\subsection*{Vigenerjeva šifra}
\[ \mathcal{B} = \mathcal{C} = \mathcal{K} = \mathbb{Z}_{25}^n\]
Ključ $\underline{k} \in \mathcal{K}$
\[ K_{\underline{k}}(\underline{x}) = \underline{x} + \underline{k} \mod 25\]
\[ D_{\underline{k}}(\underline{y}) = \underline{y} - \underline{k} \mod 25\]

\subsection*{Permutacijska šifra}
\textit{Simbolov ne nadomeščamo, ampak jih premešamo}
\[ \mathcal{B} = \mathcal{C} = \mathbb{Z}_{25}^n, \quad \mathcal{K} = S_n\]
\[ K_{\pi}(\underline{x}) = \underline{x}_{\pi(1)} + \dots + \underline{x}_{\pi(n)}  \]
\[ D_{\pi}(\underline{x}) = \underline{x}_{\pi^{-1}(1)} + \dots + \underline{x}_{\pi^{-1}(n)}  \]

\subsection*{Hillova šifra}
\[ \mathcal{B} = \mathcal{C} = \mathbb{Z}_{25}^n, \quad \mathcal{K} = \{ A \in \mathbb{Z}_{25}^{n\times n} | \det(A) \in \mathbb{Z}_{25}^* \}\]
Ključ je matrika $A \in \mathcal{K}$
\[ K_A(\underline{x}) = A \underline{x} \mod 25\]
\[ D_A(\underline{y}) = A^{-1} \underline{y} \mod 25\]

\section*{Bločne šifre}
Kripotsistem $(\mathcal{B}, \mathcal{C}, \mathcal{K}, \mathcal{E}, \mathcal{D})$ je bločna šifra dolžine n, 
če je $\mathcal{B} = \mathcal{C} = \Sigma^n$, kjer je $\Sigma$ končna abeceda.

Vsaka kodirna funkcija je ekvivalentna neki permutaciji $\Sigma^n$, njena dekodirna funkcija pa inverzu te permutacije.

\subsection*{Afina bločna šifra}
\[ \Sigma = \mathbb{Z}_m \]
\[ \mathcal{K} = \left\{ (A, \underline{b});\ A \in \mathbb{Z}_m^{n\times n}, \det(A) \in \mathbb{Z}_m^*, \underline{b} \in \mathbb{Z}^n_m \right\}  \]
\begin{align*}
	E_{(A, \underline{b})}(\underline{x}) &\equiv A \underline{x} + \underline{b} \mod m \\
	D_{(A, \underline{b})}(\underline{x}) &\equiv A^{-1} \underline{x} - \underline{b} \mod m 
\end{align*}

\section*{Teorija števil}

\subsection{Eulerjeva funkcija}
Eulerjeva funkcija nam pove koliko je obrnlivih elementov v $\mathbb{Z}_m$.

\[ | \mathbb{Z}_m^* | = \varphi(m) \]

Za $n \in \mathbb{N}$ s paraštevilskim razcepom \\ $ n = p_1^{\alpha_1} \cdot ... \cdot p_m^{\alpha_m}$ velja:
\[\varphi(n) = \varphi(p_1^{\alpha_1}) \cdot ... \cdot \varphi(p_m^{\alpha_m}) = n \prod_{ p_k \in \mathbb{P}} \left(1-\frac{1}{p_k} \right) \]

\textbf{Euljerjev izrek:}
\[\textrm{gcd}(a, m) = 1 \Leftrightarrow a^{\varphi(m)} \equiv_m 1; a \in \mathbb{Z}_m^*\]
\[a,m \in \mathbb{N} \wedge \textrm{gcd}(a, m) = 1 \Rightarrow a^{\varphi(m)} \equiv_m 1\]
\[a^{\varphi(m)} = 1 \text{ v } \mathbb{Z}_m^*\]

\textbf{Mali Fermatov izrek:} če je $m \in \mathbb{P}$ ($\varphi(m) = m-1$) in $\textrm{gcd}(a,m) = 1$, potem:
\[a^{m-1} \equiv_m 1\]

\subsubsection*{Razširjen evklidov algoritem}

\begin{koda}
vhod: $(a, b)$
($r_0$, $x_0$, $y_0$) = ($a$, 1, 0)
($r_1$, $x_1$, $y_1$) = ($b$, 0, 1)
$i$ = 1

dokler $r_i$ $\neq$ 0:
    $i$ = $i$+1
    $k_i$ = $r_{i-2} // r_{i-1}$
    $(r_i, x_i, y_i)$ = $(r_{i-2}, x_{i-2}, y_{i-2}) - k_i(r_{i-1}, x_{i-1}, y_{i-1})$
konec zanke
vrni: $(r_{i-1}, x_{i-1}, y_{i-1})$
\end{koda}

Naj bosta $a, b \in \mathbb{Z}$. Tedaj trojica $(d, x, y)$, ki jo vrne razširjen evklidov algoritem z vhodnim podatkomk $(a, b)$, zadošča:
\[ax + by = d \text{ in } d = \textrm{gcd}(a, b)\] 



\end{multicols}
\end{document}